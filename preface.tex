En este trabajo de fin de grado se explora la idea de aplicar técnicas sofisticadas de visión computacional en un ámbito tan común como es el deporte. En particular, se investiga la posibilidad de automatizar la edición de vídeo de un partido de fútbol grabado con una cámara de 360 grados desde la perspectiva del portero.

Se ha decidido particularizar en este caso debido a que el proceso de edición de los más de 90 minutos que tiene un partido de fútbol para un portero resulta muy tedioso, pues este puede pasar la mayor parte del juego inactivo, teniendo solo unas pocas intervenciones clave que son difíciles de encontrar sin revisar todo el partido de forma manual.

Se ha optado por una cámara de 360 grados ya que permite capturar una buena parte del campo y dar suficiente información sobre el partido para permitir a los modelos entender qué está pasando en cada momento. Sin embargo, el uso de este tipo de cámaras presenta una serie de dificultades adicionales que incrementan la capacidad computacional requerida de los algoritmos empleados e impiden la extracción de información concluyente a través de modelos de visión computacional. La cámara utilizada en particular es el modelo ONE X2 de Insta360.

Por último, se ha desarrollado un pipeline para transformar los videos de 360 grados en videos ya editados de imagen plana con la relación de aspecto deseada. Esta transformación encapsula la conversión del formato propietario de Insta360 a un formato público (mp4), la conversión de imágenes equirectangulares a múltiples perspectivas, el procesamiento en paralelo de cada perspectiva para la detección de objetos y la creación de una perspectiva nueva para cada imagen equirectangular con la información más relevante.
