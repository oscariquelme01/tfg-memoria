Este trabajo presenta el desarrollo de un sistema automatizado para la edición de videos de fútbol capturados con cámara de 360° desde la perspectiva del portero.

Se han utilizado técnicas de transformación de imágenes equirectangulares a planas y viceversa a través de trigonometría avanzada, detección de imágenes usando los modelos públicos de la compañía Ultralytics, YOLO (You Only Look Once) y técnicas de smoothing de vídeo a través de interpolación entre distintos frames.

El sistema logró detectar y seguir al portero con precisión en diversos escenarios presentados como tiros, centros, manos a mano... Las principales dificultades encontradas fueron el tratamiento de imágenes equirectangulares y los frames incorrectos esporádicos debido a falsos positivos en la detección de imágenes, lo cual llevaba a problemas de flickering de video. El pipeline desarrollado demostró ciertas limitaciones al transformar videos de 90+ minutos equirectangulares en clips editados de pocos minutos con buena precisión, puesto que trabajar con este tipo de imágenes es computacionalmente costoso y requiere mejor hardware del que se disponía. Además, la paralelización de tareas usando una unidad de procesamiento gráfico resultó insuficiente para acelerar lo suficiente el proceso. Sin embargo, en videos de acciones cortas el sistema detecta con precisión al portero y es capaz de dar resultados informativos sobre la acción procesada. Los resultados sugieren que a través de técnicas de optimización para evitar procesar frames en los que claramente no está pasando nada, utilizando por ejemplo canales de audio o menos vistas, se podría lograr un sistema eficiente que automatice la tarea de revisar el desempeño de un portero en el campo.

\palabrasclave{Visión computacional, fútbol, automatización, edición de video, detección de objetos}
