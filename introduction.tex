Este capítulo inicial presenta una introducción general al trabajo, exponiendo los motivos que justifican su desarrollo, los objetivos propuestos y la organización del documento.

\section[Motivación]{Motivación}
En el ambito profesional del deporte, especialmente en la elite de aquellos tan populares como el futbol, los deportistas cuentan con herramientas tecnológicas sofisticadas que les ayudan a comprender sus debilidades y fortalezas. Normalmente disponen de vídeos y 
análisis integrales hechos por profesionales del deporte.

Sin embargo, esto no es cierto para el deportista recreacional que busca mejorar sus aptitudes, obtener feedback de calidad o simplemente verse realizando su ejercicio. Típiciamente vemos a estos deportistas recurrir a su teléfono para grabar su actividad o incluso utilizar una cámara dedicada para luego pasar horas editando sus vídeos a mano, algo en lo que por norma general no suelen estar muy versados.

La aplicación práctica de estas técnicas de grabación manual presenta desafíos significativos que limitan su adopción. Los principales obstáculos incluyen el tiempo excesivo requerido para revisar, editar y recortar los momentos relevantes, así como las limitaciones del campo de visión que impiden capturar información contextual completa de las jugadas. Aunque existen casos de deportistas recreacionales que obtienen buenos resultados utilizando cámaras con lente de pez y edición manual intensiva, la inversión temporal requerida (múltiples horas por sesión) resulta insostenible para la mayoría de usuarios. Esta problemática evidencia la necesidad de desarrollar una solución tecnológica automatizada que replique con precisión el trabajo de edición manual, pero sin la inversión temporal asociada.

Particularizando en mi caso personal, como portero recreacional de distintos equipos de futbol 11 en ligas locales, el proceso de grabar, editar, recortar y producir vídeos que sean de ayuda para identificar errores potenciales y aspectos a mejorar resulta inviable debido a la cantidad de horas requeridas de trabajo manual.

Hoy en dia, gracias a los avances en visión computacional tanto a nivel de eficiencia como a nivel de resultados, se posibilita una solución plausible para aquel que busque grabarse practicando su deporte. A través de técnicas de detección de imágenes, edición de vídeo procedural y codificación de situaciones espécificas de un deporte, se pueden lograr resultados comparables a la edición manual sin necesidad de que el deportista tenga que hacer una inversión significativa de tiempo o dinero.

\section[Objetivos]{Objetivos}
Ando bien objetivo

\section[Estructura del documento]{Estructura del documento}
Ando bien estructurado
