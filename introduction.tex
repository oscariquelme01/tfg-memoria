Este capítulo inicial presenta una introducción general al trabajo, exponiendo los motivos que justifican su desarrollo, los objetivos propuestos y la organización del documento.

\section[Motivación]{Motivación}
En el ambito profesional del deporte, especialmente en la elite de aquellos tan populares como el futbol, los deportistas cuentan con herramientas tecnológicas sofisticadas que les ayudan a comprender sus debilidades y fortalezas. Normalmente disponen de vídeos y 
análisis integrales hechos por profesionales del deporte.

Sin embargo, esto no es cierto para el deportista recreacional que busca mejorar sus aptitudes, obtener feedback de calidad o simplemente verse realizando su ejercicio. Típiciamente vemos a estos deportistas recurrir a su teléfono para grabar su actividad o incluso utilizar una cámara dedicada para luego pasar horas editando sus vídeos a mano, algo en lo que por norma general no suelen estar muy versados.

La aplicación práctica de estas técnicas de grabación manual presenta desafíos significativos que limitan su adopción. Los principales obstáculos incluyen el tiempo excesivo requerido para revisar, editar y recortar los momentos relevantes, así como las limitaciones del campo de visión que impiden capturar información contextual completa de las jugadas. Aunque existen casos de deportistas recreacionales que obtienen buenos resultados utilizando cámaras con lente de pez y edición manual intensiva, la inversión temporal requerida (múltiples horas por sesión) resulta insostenible para la mayoría de usuarios. Esta problemática evidencia la necesidad de desarrollar una solución tecnológica automatizada que replique con precisión el trabajo de edición manual, pero sin la inversión temporal asociada.

Particularizando en mi caso personal, como portero recreacional de distintos equipos de futbol 11 en ligas locales, el proceso de grabar, editar, recortar y producir vídeos que sean de ayuda para identificar errores potenciales y aspectos a mejorar resulta inviable debido a la cantidad de horas requeridas de trabajo manual.

Hoy en dia, gracias a los avances en visión computacional tanto a nivel de eficiencia como a nivel de resultados, se posibilita una solución plausible para aquel que busque grabarse practicando su deporte. A través de técnicas de detección de imágenes, edición de vídeo procedural y codificación de situaciones espécificas de un deporte, se pueden lograr resultados comparables a la edición manual sin necesidad de que el deportista tenga que hacer una inversión significativa de tiempo o dinero.

\section[Objetivos]{Objetivos}
El objetivo final del proyecto es desarrollar un sistema que permita al portero recreacional instalar una cámara de lente de pez en su portería. Lograr este objetivo puede servir como caso precedente para el estudio de la automatización de obtención de videos relevantes dentro del ambito deportivo casual.

Como caso particular, se trabajará con la camara ONE X2 de la compañia insta360. Esta dispone de un SDK privado al que podemos aplicar con fines académicos y comerciales. Es necesario utilizar dicho SDK puesto que la cámara graba en un formato propietario llamado \textit{.insv}. Para manejar los videos grabados utilizando librerías públicas, primero deberemos realizar la conversión a un formato público usando la librería de insta360.

Se tratará de trabajar en la medida de lo posible con imagenes planas puesto que los modelos públicos de visión computacional más usados están entrenados con imagenes planas. Es decir, si intentamos usar la proyección de una imagen esférica la distorsión sufrida en el borde inferior y superior de la imagen harán mas dificil la detección de objetos para dichos modelos.

Con esto en mente, se definen los siguientes objetivos:

\begin{enumerate}
	\item Desarrollar un programa que permita convertir el video grabado a un formato público con el que podremos interactuar más adelante con librerías de video. El programa deberá asegurarse de mantener los metadatos necesarios y propiedades asociadas a un video de 360 grados.

	\item Implementar la funcionalidad que permita describir una imagen esférica como la combinación de las 6 perspectivas cúbicas usando trigonometría avanzada. Para ello, primero se deberá convertir la imagen de 360 grados a su proyección equirectangular.
	\item Categorizar los objetos detectados en cada una de las 6 perspectivas usando un modelo de deteccíon de clases previamente entrenado. Esta información recogida será utilizada como input del algoritmo desarrollado para calcular el siguiente frame del video resultado.
	
	\item Desarrollar un algoritmo sencillo que permitirá al programa tomar decisiones sobre a donde se debe enfocar el siguiente frame. Este algoritmo tendrá en cuenta la posición del portero, la distancia estimada, el angúlo del frame anterior para prevenir el efecto de flickering y la posición de la pelota entre otros factores.
		
	  
\end{enumerate}

Ademas de todo esto, se tratará de hacer el algoritmo lo más eficiente posible, haciendo uso de la paralelización de hilos y una unidad de procesamiento gráfica en caso de que este disponible en el ordenador en el que se ejecute el programa.

Estos objetivos buscan realizar una primera aproximación a resolver el problema y no pretenden dar una solución igual de precisa que la edición manual. Más adelante se detallará el trabajo futuro que podría realizarse para obtener resultados que se acerquen más a la calidad de un trabajo hecho a mano.


\section[Estructura del documento]{Estructura del documento}

Este documento se estructura de la siguiente forma:

\begin{description}
	\item [Capítulo 1: Introducción] Se hace una presentación inicial del proyecto, justificando la razón y motivación del desarrollo. Se definen de forma clara los objetivos y el alcance del proyecto y se muestra la estructura que seguirá el documento explicando brevemente cada cápitulo.
	\item [Capítulo 2: Estado del Arte] Se realiza un estudio sobre el estado actual de la detección de imagenes usando modelos de visión computacional. También se realiza un trabajo de investigación sobre los diversos modelos disponibles y si estos han sido previamente aplicados a la detección de objetos en el ambito deportivo.
	\item [Capítulo 3: Diseño] Se detalla brevemente el diseño del sistema de forma conceptual y se justifica que tecnologías han sido escogidas.
	\item [Capítulo 4: Implementación] Se expone de forma muy detallada el proceso de la implementación de los distintos componentes del sistema asi como las dificultades encontradas y las diversas utilidades desarrolladas que han servido de apoyo para llegar a la conclusión.
	\item [Capítulo 5: Resultados] Se presentan los resultados obtenidos en una serie de casos de prueba grabados para comprobar el funcionamiento del algoritmo
	\item [Capítulo 6: Conclusiones y Trabajo Futuro] Se realiza una reflexión sobre el trabajo realizado, la correctitud de las decisiones que se han tomado y el trabajo futuro que se puede realizar para obtener mejores resultados
\end{description}
